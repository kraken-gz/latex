\documentclass[12pt,a4paper]{article}
\usepackage[serbian]{babel}
\begin{document}
\title{Vježba broj 2}
\author{Igor Lerinc}
\maketitle
\tableofcontents

\section{Naslov prve lekcije}
Primjer \LaTeX\ dokumenta sa našim slovima ščćžđ ŠČĆŽĐ. U dokumentu se
paragrafi odvajaju sa jednom \emph{praznom} linijom

{\Large Dio dokumenta može biti prikazan velikim slovima. \par }
Današnji datum je: \today

\section{Korišćenje naslova}
Tekst prvog odjeljka dokumenta. Provjerite izgled dokumenta sa opcijama:
jedna kolona a zatim sa tekstom veličine 10pt. Na kraju promijenite klasu
dokumenta u \mbox{\textbf{report}}.
\subsection{Podnaslov}\label{abc}
\subsection{još jedan podnaslov}



\section[Novi naslov]{Novi naslov koji je jako dugačak}
 Tekst u okviru podnaslova. Uočite da podnaslov ima oznaku.
\section[Novi naslov]{Novi naslov koji je jako dugačak2}
U tekstu je referenca na prethodni podnaslov čija je oznaka: \ref{abc} i
koji se
nalazi na strani: \pageref{abc}.
 
 
 
 
\end{document}
 
 