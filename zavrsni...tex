 \documentclass[12pt,a4paper, twocolumn] {article}
 \usepackage[utf8]{inputenc}
 \usepackage[serbian]{babel}
 \usepackage{amsmath,amssymb}
\begin{document}
\title{Završni ispit - popravni termin}
\author{Igor Lerinc 14/21}
\date{17. decembar 2021}
\maketitle


\section{Tekst}
Kažete da ste neshvaćeni mladi prijatelju. Strpite se. I dijamant je u osnovi ugljenik, ali je umio da čeka. \footnote{Pablo Pikaso}

\textit{Ako si se uputio prema cilju, i putem i putem počeo da sastajkuješ i kamenjem  gađaš svakog psa koji na tebe laje, nikada nećeš stići na cilj.} \footnote{F.M. Dostojevski}

Nakon uploada PDF fajla unesite polje za odgovor: %\mathbf{2196}

\section{Formule}
Nejednakost $ (x+3)^2 \geq 50$
Suma i zagrade promenjive veličine
Formula je numerisana

\
$ \beta = \bigg(\sum_{n=1}^{N} \frac{1}{n^2 +n}\bigg)^2 $


\subsection{Formula u više redova}
Koristite okruženje \textbf{align}
$$ 64x + 6y = 0 $$
$$ 36x - 2y + 54z = 53 $$
$$ 5y=3 $$

\section{tabela i lista}
\medskip
Tabela je centrirana (nalazi se u center okruženju).
Poslije tabele je numerisana lista.


\begin{center}
\begin{tabular}{|c|c|c|}
\hline
\textbf{Ime} & \textbf{Poeni} & \textbf{Ocjena} \\
\hline
Mileva & 95 & A\\
\hline
Luka & 71 & C\\
\hline
Jelica & 95 & A \\
\hline
Vasilije & 65 & D\\
\hline
\end{tabular}
\end{center}


\begin{enumerate}
\item Mileva i Jelica su dobili ocjenu \textbf{A}
\item Luka ima ocjenu \textbf{C}
\item Vasilije ima ocjenu \textbf{D}


\end{enumerate}



\tableofcontents








\end{document} 
 
 
 
 
 
 
 
 