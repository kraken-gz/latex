 \documentclass[a4paper, 12pt]{article}
 \begin{document}
 
 

 \section{Naslov sekcije}
Novu sekciju počinješ sa komandom \backslash section{}

\subsection{Naslov podsekcije}
 Podsekciju počinješ sa \backslash subsection{}
 
\subsubsection{Naslov pod-podsekcije}
 Naredni nivo podsekcije počinješ sa \backslash subsubsection{}
 
 
 - U dokumentima klase report ili book imamo i odeljke višeg nivoa , kao što je:
 poglavlje \backslash chapter{} i dio: \backslash part{}
 
 \chapter{Poglavlje (glava) 1}
 \part{Dio II}
 
 
 
 - Sve navedene komande imaju svoju verziju sa zvjezdicom npr.
 
 \section*{Zaključak}
 
 U tom slučaju se ne vrši numeracija definisanog odeljka dokumenta (ne računa stranice, te stranice koje su, sa zvezdicom..)
 
 
 \chapter[Tekst]{Tekst -- unos, korekcija i ispis}
 
 - Sadrzaj u dokument umećemo naredbom
 \tableofcontents


- Sadrzaj se moze uneti na proizvoljno mesto u dokumentu, ali obično je on na početku ili na kraj dokumenta

- Bilo koji numerisani dio dokumenta, mozemo oznaciti komandom  
\label{ImeOznake}
Argument ove naredbe je ime oznake i treba biti jedinstveno na nivou dokumenta 

- Pozivanje na redni broj oznake ostvarujemo komandom  \ref{ImeOznake}
Znaci, koji je onaj redni broj te oznake, 2.1, 2.1.3 itd..

- Stranicu na kojoj smo postavili oznaku dobijamo komandom \pageref{ImeOznake}

-Obe komande, \ref i \pageref kao argument uzimaju ima koje smo dodelili numerisanom objaktu komandom \label



 
 
 
 
 \end{document}